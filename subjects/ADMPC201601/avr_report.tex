%%%%%%%%%%%%%%%%%%%%%%%%%%%%%%%%%%%%%%%%%
% Beamer Presentation
% LaTeX Template
% Version 1.0 (10/11/12)
%
% This template has been downloaded from:
% http://www.LaTeXTemplates.com
%
% License:
% CC BY-NC-SA 3.0 (http://creativecommons.org/licenses/by-nc-sa/3.0/)
%
%%%%%%%%%%%%%%%%%%%%%%%%%%%%%%%%%%%%%%%%%

%----------------------------------------------------------------------------------------
%	PACKAGES AND THEMES
%----------------------------------------------------------------------------------------

\documentclass{beamer}

\mode<presentation> {

\usetheme{Madrid}
\usecolortheme{dolphin}
%\setbeamertemplate{footline} % To remove the footer line in all slides uncomment this line
%\setbeamertemplate{footline}[page number] % To replace the footer line in all slides with a simple slide count uncomment this line

%\setbeamertemplate{navigation symbols}{} % To remove the navigation symbols from the bottom of all slides uncomment this line
}

\usepackage{graphicx} % Allows including images
\usepackage{ragged2e} % Jusitfy
\usepackage{booktabs} % Allows the use of \toprule, \midrule and \bottomrule in tables
\usepackage{lmodern}
\usepackage{amsfonts}
\usepackage{amsmath}
\usepackage{mathtools}
\usepackage{multicol}
\usepackage{listings} % C++ code
\lstset{language=C++,
                basicstyle=\footnotesize\ttfamily,
                keywordstyle=\footnotesize\color{blue}\ttfamily,
}
\addtobeamertemplate{block begin}{}{\justifying} %Justify
%----------------------------------------------------------------------------------------
%	TITLE PAGE
%----------------------------------------------------------------------------------------

\title[AVR Set Instructions Analysis]{AVR Set Instructions Analysis} % The short title appears at the bottom of every slide, the full title is only on the title page

\author{Ulises M\'endez Mart\'{i}nez} % Your name
\institute[UAG] % Your institution as it will appear on the bottom of every slide, may be shorthand to save space
{
Universidad Aut\'onoma de Guadalajara  \\ % Your institution for the title page
\medskip
\textit{ulisesmdzmtz@gmail.com} % Your email address
}
\date{\today} % Date, can be changed to a custom date

\begin{document}

\begin{frame}
\titlepage % Print the title page as the first slide
\end{frame}


%--------------------------------------------------------
%	PRESENTATION SLIDES
%--------------------------------------------------------

%--------------------------------------------------------
\section{Introduction} 
\begin{frame}
\frametitle{Introduction}
Most are single 16 bit words; four marked * have a second word
to define an address or address extension ($kkkk kkkk kkkk kkkk$)
\\
\begin{itemize}
\item  d  bits that specify an Rd (0..31 or 16..31 or 16..23 or 24..30)\\
\item  r  bits that specify an Rr ( - ditto - )\\
\item  k  bits that specify a constant or an address\\
\item  q  bits that specify an offset\\
\item  -  bit that specifies pre-decrement mode: 0=no, 1=yes\\
\item  +  bit that specifies post-decrement mode: 0=no, 1=yes\\
\item  x  bit of any value\\
\item  s  bits that specify a status bit (0..7)\\
\item  A  bits that specify i/o memory\\
\item  b  bits that define a bit (0..7)\\
\end{itemize}
\end{frame}
%--------------------------------------------------------
\section{Instruction Summary} 
\subsection{Blocks of 256 instructions}
\begin{frame}
\frametitle{Summary Table I}
\small
\begin{center}
    \begin{tabular}{| l | l | l | l | l | l | l | l | l |}
    \hline
    - & x0zz & x1zz & x2zz & x3zz & x4zz & x5zz & x6zz & x7zz \\    \hline
    0xzz & [R] nop & movw & muls & fmul & cpc & cpc & cpc  & cpc \\ \hline
    1xzz & cpse & cpse & cpse & cpse & cp & cp & cp & cp \\ \hline
    2xzz & and & and & and & and & eor & eor & eor & eor \\ \hline
    3xzz & cpi & cpi & cpi & cpi & cpi & cpi & cpi & cpi \\ \hline
    4xzz & sbci & sbci & sbci & sbci & sbci & sbci & sbci & sbci \\ \hline
    5xzz & & & & & & & & \\ \hline
    6xzz & & & & & & & & \\ \hline
    7xzz & & & & & & & & \\ \hline
    8xzz & & & & & & & & \\ \hline
    9xzz & & & & & & & & \\ \hline
    Axzz & & & & & & & & \\ \hline
    Bxzz & & & & & & & & \\ \hline
    Cxzz & & & & & & & & \\ \hline
    Dxzz & & & & & & & & \\ \hline
    Exzz & & & & & & & & \\ \hline
    Fxzz & & & & & & & & \\ \hline
    \end{tabular}
\end{center}
\end{frame}
%--------------------------------------------------------
\begin{frame}
\frametitle{Summary Table II}
\small
\begin{center}
    \begin{tabular}{| l | l | l | l | l | l | l | l | l |}
    \hline
    - & x8zz & x9zz & xAzz & xBzz & xCzz & xDzz & xEzz & xFzz \\    \hline
    0xzz & & & & & & & & \\ \hline
    1xzz & & & & & & & & \\ \hline
    2xzz & & & & & & & & \\ \hline
    3xzz & & & & & & & & \\ \hline
    4xzz & & & & & & & & \\ \hline
    5xzz & & & & & & & & \\ \hline
    6xzz & & & & & & & & \\ \hline
    7xzz & & & & & & & & \\ \hline
    8xzz & & & & & & & & \\ \hline
    9xzz & & & & & & & & \\ \hline
    Axzz & & & & & & & & \\ \hline
    Bxzz & & & & & & & & \\ \hline
    Cxzz & & & & & & & & \\ \hline
    Dxzz & & & & & & & & \\ \hline
    Exzz & & & & & & & & \\ \hline
    Fxzz & & & & & & & & \\ \hline
    \end{tabular}
\end{center}
\end{frame}

%--------------------------------------------------------
\section{AVR opcodes}
\begin{frame}
\frametitle{C++}



\end{frame}
%--------------------------------------------------------

%--------------------------------------------------------
\section{Code Implementation}
\begin{frame}
\frametitle{C++}

\end{frame}
%--------------------------------------------------------

\begin{frame}[fragile]
\frametitle{ Solution using C++ }
\begin{example}[ C++ Implementation ]
\begin{lstlisting}
// Macros for bits manipulation
#define get4Bits(w,B) (((w)&((0xF)<<(B*4)))>>(B*4))
#define getByte(w,B) (((w)&((0xFF)<<(B*8)))>>(B*8))
#define getBit(w,b) (((w)&((0x1)<<b))>>b)
#define shiftBits(w,b) ((w)<<b)
\end{lstlisting}
\end{example}
\end{frame}
%--------------------------------------------------------
%--------------------------------------------------------
\begin{frame}[fragile]
\frametitle{ Solution using C++ }
\begin{example}[ C++ Implementation ]
\begin{lstlisting}
// Macros for string processing
#define clean(s) sprintf((s),"")
#define cut(s,i,sz) ((s).substr((i),(sz)))
#define to_c(s) ((s).c_str())
#define getHex(s,n) (sscanf(s,"%x",&(n)))
\end{lstlisting}
\end{example}
\end{frame}
%--------------------------------------------------------
%--------------------------------------------------------
\begin{frame}
\frametitle{ References }
\begin{itemize}
	\item http://lyons42.com/AVR/Opcodes/AVRAllOpcodes.html
	\item http://www.avrfreaks.net/forum/looking-avr-opcode-table
	\item https://en.wikipedia.org/wiki/Intel\_HEX
\end{itemize}
\end{frame}

\end{document} 
